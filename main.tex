\documentclass[12pt,a4paper,oneside]{report}

% ---- Paquetes básicos ----
\usepackage[spanish,es-noshorthands]{babel}
\usepackage{iftex}
\ifPDFTeX
  \usepackage[utf8]{inputenc}
  \usepackage[T1]{fontenc}
\else
  \usepackage{fontspec}
  \setmainfont{Latin Modern Roman}
\fi
\usepackage{geometry}
\geometry{top=2.5cm, bottom=2.5cm, left=2.5cm, right=2.5cm}
\usepackage{setspace}
\onehalfspacing
\usepackage{graphicx}
\usepackage{tikz}
\usetikzlibrary{shapes.geometric, arrows.meta, positioning}

\usepackage{forest}

\usepackage{caption}
\captionsetup{
  font=small,            % tamaño del texto del pie
  labelfont=bf,          % "Figura 1.1:" en negrita
  format=hang,           % alinear líneas tras la etiqueta
  justification=justified,  % o 'raggedright' si prefieres bandera izquierda
  singlelinecheck=true, % que no cambie a centrado en una sola línea
  width=\linewidth,      % el pie ocupa el ancho del texto
  skip=6pt,              % espacio entre figura y pie
  margin=0pt             % sin márgenes extra laterales en el caption
}
\usepackage{subcaption}
\usepackage{booktabs}
\usepackage{float}

\newcommand{\pendiente}{\noindent\textit{(Contenido pendiente.)}\par}
\usepackage{amsmath,amssymb,amsthm}
\usepackage{csquotes}
\usepackage{xcolor}
\usepackage{enumitem}
\usepackage{hyperref}
\hypersetup{
  colorlinks=true,
  linkcolor=blue,
  citecolor=blue,
  urlcolor=blue,
  pdfauthor={Álvaro Viña Pérez},
  pdftitle={TFG - Estructura oficial}
}
\usepackage{cleveref}
\usepackage{listings}
\lstset{basicstyle=\ttfamily\small,breaklines=true,frame=single}
% (Opcional) Minted para código con resaltado: requiere -shell-escape
% \usepackage{minted}

% ---- Bibliografía ----
\usepackage[backend=biber,style=numeric,sorting=nyt]{biblatex}
\addbibresource{library.bib} % crea un refs.bib en el proyecto

% ---- Glosario y acrónimos (opcional) ----
\usepackage[acronym,nonumberlist]{glossaries}
\makenoidxglossaries
\setglossarysection{subsubsection}

% Crear glosario de definiciones
\newglossaryentry{robot}{
    name={Robot},
    text={robot},
    plural={robots},
    description={Máquina programable capaz de realizar tareas de forma autónoma o semiautónoma, generalmente mediante la ejecución de movimientos precisos y repetitivos.}
}
\newglossaryentry{cinematica}{
    name={Cinemática},
    description={Rama de la mecánica que estudia el movimiento de los cuerpos sin considerar las fuerzas que lo producen. En robótica, se refiere al análisis del movimiento de los manipuladores y sus eslabones.}
}

\newglossaryentry{redneuronal}{
    name={Red Neuronal},
    description={Modelo computacional inspirado en la estructura y funcionamiento del cerebro humano, compuesto por nodos (neuronas) interconectados que procesan información y aprenden a partir de datos.}
}
\newglossaryentry{matlab}{
    name={Matlab},
    description={Entorno de programación y cálculo numérico desarrollado por MathWorks, ampliamente utilizado en ingeniería y ciencias para el análisis de datos, simulación y desarrollo de algoritmos.}
}
\newglossaryentry{python}{
    name={Python},
    description={Lenguaje de programación de alto nivel, interpretado y de propósito general, conocido por su sintaxis clara y legibilidad, utilizado en una amplia variedad de aplicaciones, desde desarrollo web hasta análisis de datos y aprendizaje automático.}
}


% Definición de acrónimos
\newacronym{IDeTIC}{IDeTIC}{Instituto para el Desarrollo Tecnológico y la Innovación en Comunicaciones}
\newacronym{HT}{HT}{Humano-Tableta}
\newacronym{R}{R}{Robot}
\newacronym{RT}{RT}{Robot-Tableta}
\newacronym{WIPL}{WIPL}{Wacom Intuos Pro Large}
\newacronym{IOTP}{IOTP}{Wacom IOT Paper 3}
\newacronym{UR5}{UR5}{Universal Robots 5}
\newacronym{IDE}{IDE}{Entorno de Desarrollo Integrado (\textit{Integrated Development Environment})}
\newacronym{CSV}{CSV}{Archivo de valores separados por comas (\textit{Comma-Separated Values})}
\newacronym{HTR}{HTR}{Error entre el movimiento humano y el ejecutado por el robot}
\newacronym{RRT}{RRT}{Error entre el movimiento ejecutado por el robot y el registrado por la tableta}
\newacronym{HTRT}{HTRT}{Error entre el movimiento humano y el registrado por la tableta}
\newacronym{RMSE}{RMSE}{Root Mean Square Error}
\newacronym{MAE}{MAE}{Mean Absolute Error}
\newacronym{MSE}{MSE}{Mean Squared Error}
\newacronym{MTP}{MTP}{Media Transfer Protocol}



% ---- Comandos útiles ----
\newcommand{\todo}[1]{\textcolor{red}{[TODO: #1]}}
\newcommand{\figref}[1]{\Cref{#1}}
\newcommand{\tabref}[1]{\Cref{#1}}

\begin{document}

% ============ ÍNDICE ============
\pagenumbering{roman}
\tableofcontents
\clearpage
\listoffigures
\clearpage
\listoftables
\clearpage
\pagenumbering{arabic}

% ==============================================
% DOCUMENTO I: MEMORIA DESCRIPTIVA
% (Estructura prioritaria según tu solicitud)
% ==============================================
\chapter{Documento I: Memoria Descriptiva}
\clearpage

\section{Información previa}\label{sec:info-previa}

\subsection{Introducción}


En la actualidad, los sistemas de digitalización han adquirido una gran relevancia en numerosos ámbitos científicos y tecnológicos. 
Dispositivos como las tabletas gráficas permiten registrar de manera precisa trazos y movimientos, lo que abre la puerta a múltiples 
aplicaciones que van desde el diseño asistido hasta el análisis del comportamiento humano. Sin embargo, todo instrumento de medida 
presenta un margen de error, y conocer dicho error resulta esencial para valorar la fiabilidad de los datos recogidos y garantizar 
la validez de los estudios que se apoyan en ellos.  

El análisis de la escritura constituye un campo especialmente interesante dentro de este marco, ya que la forma en la que una persona 
escribe puede ofrecer información sobre sus capacidades motoras, cognitivas o incluso sobre posibles alteraciones neurológicas. 
Por ello, disponer de un sistema que permita comparar la escritura humana con la reproducida por un robot o capturada por distintos 
dispositivos digitalizadores ofrece una oportunidad única para estudiar cómo se generan los trazos y qué diferencias se introducen 
por efecto del medio utilizado.  

Además de su interés puramente técnico, la posibilidad de caracterizar y cuantificar estos errores abre nuevas líneas de aplicación 
en el desarrollo de herramientas basadas en inteligencia artificial. Sistemas entrenados con este tipo de datos pueden contribuir 
a la detección temprana de enfermedades, al análisis del envejecimiento motor o a la mejora de interfaces hombre-máquina en el ámbito 
médico y educativo. Así, el estudio de los errores de digitalización no debe entenderse únicamente como un ejercicio académico, 
sino como un paso necesario hacia el aprovechamiento de la escritura como fuente de información biomédica.  

El presente trabajo se centra en la comparación de distintos métodos de registro y reproducción de la escritura, empleando tanto la 
acción directa de un usuario sobre una tableta como la ejecución de trazos mediante un robot colaborativo. El proceso seguido consiste, 
a grandes rasgos, en capturar un conjunto de palabras, reproducirlas en diferentes sistemas y analizar las diferencias existentes 
entre los resultados. A partir de dichas comparaciones es posible estimar el error asociado a cada dispositivo y obtener conclusiones 
sobre la fiabilidad de los datos que generan.  

En definitiva, este estudio pretende aportar una visión global sobre la importancia de conocer las limitaciones de los sistemas de 
digitalización, destacando su potencial como herramienta para el análisis de la escritura y su aplicación futura en el entrenamiento 
de algoritmos de inteligencia artificial destinados al ámbito médico y al desarrollo de tecnologías de apoyo.  

\subsection{Peticionario}

El presente proyecto ha sido solicitado por el \acrfull{IDeTIC}, en el marco de las actividades académicas
vinculadas al Grado en Ingeniería Electrónica Industrial y Automática de la Escuela de
Ingenieros Industriales y Civiles de la Universidad de Las Palmas de Gran Canaria.

\subsection{Antecedentes}

La tecnología de captura de movimiento permite registrar movimientos físicos
y digitalizarlos con el fin de analizarlos o replicarlos en entornos digitales.
A pesar de los avances alcanzados en los últimos años, estos sistemas todavía
enfrentan un reto importante: su precisión y fiabilidad siguen siendo
insuficientes para reproducir con exactitud los movimientos humanos mediante
un brazo robótico.  

El uso de técnicas de captura de movimiento no es reciente. Sus primeras
aplicaciones se dieron en el ámbito de la animación digital y la industria del
entretenimiento, donde se empleaban para trasladar los gestos de actores a
personajes virtuales. Con el paso del tiempo, la tecnología se ha extendido a
sectores tan diversos como la ingeniería, la medicina, el deporte y la robótica,
en los que la necesidad de registrar trayectorias con precisión ha cobrado
mayor relevancia.  

Los instrumentos de digitalización de movimiento, como las tabletas gráficas
y los sensores de movimiento tridimensional (3D), se han consolidado como
herramientas fundamentales en ámbitos como la robótica, la animación digital
y la medicina. Estos dispositivos proporcionan a los profesionales la capacidad
de capturar, manipular y analizar movimientos de manera más eficiente y
precisa que mediante técnicas tradicionales.  

En particular, las tabletas gráficas permiten registrar movimientos en dos
dimensiones. Son dispositivos relativamente accesibles y fáciles de utilizar,
aunque su nivel de precisión depende en gran medida de la calidad y la gama
del producto. Por otra parte, los sensores de movimiento 3D ofrecen la
posibilidad de capturar desplazamientos en un espacio tridimensional,
aportando una representación más realista y completa, si bien presentan un
coste de adquisición elevado y una mayor complejidad técnica en su uso e
integración.  

Más allá de las diferencias entre dispositivos, existen limitaciones comunes en
este tipo de tecnologías: la necesidad de calibración frecuente, la influencia
del ruido en la señal, la latencia en la transmisión de datos y la dependencia
del hardware utilizado. Estas restricciones pueden comprometer la fiabilidad
del sistema, especialmente en aplicaciones donde se requiere alta precisión,
como en la interacción humano–robot o la programación automática de
trayectorias.  

En este contexto, el presente proyecto plantea un análisis comparativo de los
errores generados por diferentes instrumentos digitalizadores. El propósito es
comprender cómo dichos errores afectan a la precisión y fiabilidad de los
sistemas de captura, ya que estos aspectos son determinantes para el correcto
funcionamiento de aplicaciones en las que la exactitud del movimiento resulta
crítica, como la interacción humano–robot, la rehabilitación asistida por
robot o la simulación de procesos en entornos virtuales.

\subsection{Ubicación}
El proyecto se llevó a cabo en el \todo{Nombre del laboratorio} del IDeTIC. 
En la \figref{fig:UbicacionCampus} se muestra la ubicación del pabellon donde 
se situa el laboratorio dentro del campus universitario, 
mientras que en la \figref{fig:UbicacionLP} se indica su 
localización en el municipio de Las Palmas de Gran Canaria.

\begin{figure}[h]
  \begin{subfigure}[b]{0,4\textwidth}
    \centering
    \includegraphics[height=4.2cm,keepaspectratio]{figuras/ubicacionCampus.png}
    \subcaption{Ubicación del laboratorio dentro del campus universitario.}
    \label{fig:UbicacionCampus}  
  \end{subfigure}
  \hfill
  \begin{subfigure}[b]{0,4\textwidth}
    \centering
    \includegraphics[height=4.2cm,keepaspectratio]{figuras/ubicacionLP.png}
    \subcaption{Ubicación del laboratorio en el municipio de Las Palmas.}
    \label{fig:UbicacionLP}   
  \end{subfigure}
\caption{Ubicación del laboratorio donde se desarrolló el proyecto.}
\label{fig:UbicacionLaboratorio}
\end{figure}

\subsection{Necesidades a satisfacer y justificación}

El desarrollo del presente proyecto responde a la necesidad de contar con
herramientas que permitan evaluar con rigor la precisión y fiabilidad de los
sistemas de captura de movimiento empleados en la interacción con brazos
robóticos. A partir del análisis de los antecedentes, se identifican las
siguientes necesidades principales:

\begin{itemize}
    \item \textbf{Medir objetivamente los errores de digitalización}: disponer
    de un procedimiento que cuantifique la diferencia entre el movimiento
    humano, el ejecutado por el brazo robótico y el registrado por los
    instrumentos de captura.

    \item \textbf{Comparar distintas tecnologías de captura}: analizar el
    comportamiento de tabletas gráficas de diferentes gamas frente a sensores
    tridimensionales, con el fin de establecer criterios de selección en
    función de la relación coste–prestaciones.

    \item \textbf{Mejorar la fiabilidad de la interacción humano–\glspl{robot}}:
    detectar las limitaciones de cada sistema de captura que puedan afectar a
    la ejecución precisa de trayectorias y proponer estrategias de mitigación.

    \item \textbf{Facilitar la integración en entornos académicos e
    industriales}: generar una base de conocimiento que permita seleccionar
    tecnologías de digitalización adecuadas para proyectos de investigación,
    docencia y aplicaciones prácticas en automatización y robótica.

\end{itemize}

En conjunto, estas necesidades reflejan la importancia de evaluar de forma
crítica las herramientas de captura de movimiento y su impacto en la
programación y control de \glspl{robot}, garantizando resultados reproducibles y
útiles para futuras aplicaciones en interacción humano–robot.

\subsection{Objeto del trabajo}

El objeto del presente proyecto es el estudio, caracterización y análisis de los errores producidos por distintos instrumentos digitalizadores de captura de movimiento cuando un brazo robótico ejecuta trayectorias previamente programadas.

El trabajo persigue comprender de forma detallada cómo se comportan diferentes sistemas de adquisición de movimiento frente a la ejecución real del \gls{robot} y frente al gesto humano que origina la trayectoria. Para ello, el análisis se centra en tres tipos de error principales:
\begin{enumerate}
	\item El error existente entre el movimiento realizado por el usuario y el que finalmente ejecuta el brazo robótico.
	\item El error entre el movimiento ejecutado por el brazo robótico y el que registran los instrumentos digitalizadores.
	\item El error resultante de comparar directamente el movimiento realizado por el usuario con el registrado por los instrumentos digitalizadores.
\end{enumerate}

Como instrumentos de digitalización se emplearán tabletas gráficas de 
distintas gamas, seleccionadas en función de su precio y calidad, 
con el fin de estudiar cómo estas variables influyen en la precisión 
de los datos obtenidos. Además, se incorporarán sensores de captura 
de movimiento tridimensional (3D), que permitirán contrastar y 
complementar los resultados derivados de las tabletas, ofreciendo 
una perspectiva más amplia y comparativa de las capacidades de cada 
tecnología.

Este enfoque permitirá no solo cuantificar los errores presentes en cada etapa
(Usuario–Robot, Robot–Instrumento, Usuario–Instrumento),
sino también identificar las limitaciones y fortalezas de
cada sistema de digitalización. Con ello se busca establecer
una base objetiva que facilite la selección de la tecnología
más adecuada en función de los requisitos de precisión, coste
y aplicabilidad en distintos entornos de trabajo, especialmente
en el ámbito de la interacción humano–robot y la programación de
trayectorias asistidas.

\subsection{Definiciones y abreviaturas}
\printnoidxglossary[title={Definiciones}]
\printnoidxglossary[type=\acronymtype,title={Acrónimos}]

\subsection{Marco teórico}


El marco teórico de este proyecto se fundamenta en los principios y conceptos
básicos de la robótica, en particular aquellos relacionados con la \gls{cinematica},
la dinámica y el control de manipuladores robóticos. Como referencia principal
se emplea el libro \textit{Fundamentos de Robótica} de Barrientos et al.
\cite{barrientos2014}, en el cual se abordan de manera sistemática los
aspectos esenciales de la robótica moderna.

De acuerdo con Barrientos et al. \cite{barrientos2014}, un manipulador robótico
puede describirse como un conjunto de eslabones articulados que permiten
ejecutar movimientos en el espacio tridimensional mediante la aplicación de
leyes de \gls{cinematica} y dinámica. En este contexto, la caracterización de la
posición, la orientación y la trayectoria del efector final resulta
fundamental para garantizar la precisión y repetibilidad del sistema.

Asimismo, el estudio de los sistemas de referencia, las transformaciones
homogéneas y los algoritmos de control asociados es imprescindible para
comprender cómo el \gls{robot} interpreta y ejecuta las trayectorias programadas.
Estos conceptos se aplicarán directamente en el presente proyecto para
analizar la diferencia entre el movimiento humano de referencia, el movimiento
ejecutado por el brazo robótico y el movimiento registrado por los distintos
instrumentos digitalizadores.

De este modo, el marco teórico no solo proporciona las bases matemáticas y
conceptuales de la robótica, sino que también sustenta la motivación del
proyecto: evaluar la fiabilidad y precisión de diferentes instrumentos de
captura de movimiento en la interacción con un manipulador robótico.

\subsubsection{Traslación entre sistemas de referencia}

En robótica, el uso de sistemas de referencia es fundamental para describir la
posición y el movimiento de un manipulador en el espacio. Cuando se trabaja con
más de un sistema de coordenadas, resulta necesario establecer relaciones que
permitan expresar un punto definido en un sistema en términos de otro.  

La traslación entre sistemas de referencia consiste en desplazar el origen de
un sistema con respecto a otro, manteniendo la orientación de sus ejes
paralela. Matemáticamente, si se considera un sistema de referencia $\{A\}$ y
otro sistema $\{B\}$ cuyo origen se encuentra en el vector
$\mathbf{p} = [p_x, p_y, p_z]^T$ respecto a $\{A\}$, cualquier punto
$\mathbf{r}_B$ expresado en coordenadas del sistema $\{B\}$ puede representarse
en el sistema $\{A\}$ como la ecuación~\eqref{eq:traslacion}:

\begin{equation}
\mathbf{r}_A = \mathbf{r}_B + \mathbf{p}
\label{eq:traslacion}
\end{equation}

donde $\mathbf{r}_A$ es el vector de posición del punto en el sistema $\{A\}$.
Esta expresión refleja que la traslación se implementa mediante la suma del
vector que define el desplazamiento entre los orígenes.  

En términos matriciales, la operación puede escribirse como la ecuacion~\eqref{eq:traslacion_matriz}:

\begin{equation} 
\begin{bmatrix}
x_A \\ y_A \\ z_A
\end{bmatrix}
=
\begin{bmatrix}
x_B \\ y_B \\ z_B
\end{bmatrix}
+
\begin{bmatrix}
p_x \\ p_y \\ p_z
\end{bmatrix}
\label{eq:traslacion_matriz}
\end{equation}


De esta forma, la traslación permite relacionar de manera directa dos sistemas
de referencia que comparten la misma orientación, siendo esta una herramienta
básica para la modelización de manipuladores robóticos y el cálculo de
trayectorias.  

Tal y como señala Barrientos et al. \cite{barrientos2014}, esta formulación es
el punto de partida para operaciones más complejas que incluyen además
rotaciones y, en general, transformaciones homogéneas, necesarias para
describir la \gls{cinematica} de \glspl{robot} manipuladores.

\subsubsection{Rotación entre sistemas de referencia}

Cuando dos sistemas de coordenadas comparten el mismo origen, pero presentan
una orientación diferente de sus ejes, es necesario establecer la relación que
permite expresar un vector definido en un sistema en términos del otro. Este
proceso se denomina \textbf{rotación entre sistemas de referencia} como se 
representa en la~\figref{fig:Rotacion_sistemas_referencia}.  

Sea un sistema $\{A\}$ con ejes $\hat{x}_A$, $\hat{y}_A$ y $\hat{z}_A$, y un
sistema $\{B\}$ cuyos ejes $\hat{x}_B$, $\hat{y}_B$ y $\hat{z}_B$ están
rotados respecto a $\{A\}$. La posición de un punto $\mathbf{r}$ expresada en
el sistema $\{B\}$ se puede convertir al sistema $\{A\}$ mediante una matriz de
rotación $\mathbf{R}$~\eqref{eq:rotacion_matriz}:

\begin{equation}
\mathbf{r}_A = \mathbf{R} \, \mathbf{r}_B
\label{eq:rotacion_matriz}
\end{equation}

donde $\mathbf{R}$ es una matriz ortogonal de dimensión $3 \times 3$, cuyas
columnas están formadas por las componentes de los vectores unitarios de los
ejes de $\{B\}$ expresados en el sistema $\{A\}$.  

\begin{figure}[h]
\centering
\includegraphics[width=0.6\textwidth]{figuras/rotacion2.png}
\caption{Rotación entre sistemas de referencia. De Barrientos et al. \cite{barrientos2014}.}
\label{fig:Rotacion_sistemas_referencia}
\end{figure}

De forma general, la matriz de rotación cumple las siguientes propiedades
(Barrientos et al., \cite{barrientos2014}):  
\begin{itemize}
    \item $\mathbf{R}^T \mathbf{R} = \mathbf{I}$ (ortogonalidad).  
    \item $\det(\mathbf{R}) = +1$ (preserva la orientación).  
\end{itemize}

En robótica es habitual trabajar con rotaciones elementales alrededor de los
ejes principales. Estas se representan mediante matrices específicas:  

- Rotación alrededor del eje $x$ un ángulo $\theta$:  
\begin{equation}
R_x(\theta) =
\begin{bmatrix}
1 & 0 & 0 \\
0 & \cos \theta & -\sin \theta \\
0 & \sin \theta & \cos \theta
\end{bmatrix}
\label{eq:rotacion_matrizX}
\end{equation}

- Rotación alrededor del eje $y$ un ángulo $\theta$:  
\begin{equation}
R_y(\theta) =
\begin{bmatrix}
\cos \theta & 0 & \sin \theta \\
0 & 1 & 0 \\
-\sin \theta & 0 & \cos \theta
\end{bmatrix}
\label{eq:rotacion_matrizY}
\end{equation}

- Rotación alrededor del eje $z$ un ángulo $\theta$:  
\begin{equation}
R_z(\theta) =
\begin{bmatrix}
\cos \theta & -\sin \theta & 0 \\
\sin \theta & \cos \theta & 0 \\
0 & 0 & 1
\end{bmatrix}
\label{eq:rotacion_matrizZ}
\end{equation}

Estas rotaciones básicas pueden combinarse para describir cualquier cambio de
orientación en el espacio tridimensional. El orden en que se apliquen las
rotaciones es determinante, ya que la composición no es conmutativa.  

Tal y como señala Barrientos et al. \cite{barrientos2014}, la correcta
formulación de estas transformaciones es esencial en la \gls{cinematica} de \glspl{robot},
pues permite relacionar las coordenadas de puntos y vectores entre distintos
eslabones de un manipulador articulado.


\subsubsection{Matrices de transformación homogénea}

La combinación de traslaciones y rotaciones es fundamental en la modelización
de manipuladores robóticos, ya que permite describir cómo se relacionan entre sí
los distintos sistemas de referencia asociados a cada eslabón. Para expresar
de manera compacta estas transformaciones se recurre a las
\textbf{matrices de transformación homogénea}.

Una matriz de transformación homogénea $\,\mathbf{T}\,$ es de dimensión
$4 \times 4$ y tiene la siguiente forma general:

\begin{equation}
\mathbf{T} =
\begin{bmatrix}
\mathbf{R} & \mathbf{p} \\
\mathbf{0}^T & 1
\end{bmatrix}
\label{eq:transformacion_homogenea}
\end{equation}

donde:
\begin{itemize}
    \item $\mathbf{R}$ es la matriz de rotación $3 \times 3$ que define la
    orientación de un sistema de referencia $\{B\}$ respecto a otro $\{A\}$.
    \item $\mathbf{p} = [p_x, p_y, p_z]^T$ es el vector de traslación que
    representa la posición del origen de $\{B\}$ respecto a $\{A\}$.
    \item $\mathbf{0}^T = [0 \; 0 \; 0]$ es un vector fila que completa la
    estructura de la matriz.
\end{itemize}

Si un punto $\mathbf{r}_B$ se expresa en coordenadas homogéneas en el sistema
$\{B\}$, es decir, como $\mathbf{r}_B^h = [x_B, y_B, z_B, 1]^T$, su
representación en el sistema $\{A\}$ se obtiene aplicando:

\begin{equation}
\mathbf{r}_A^h = \mathbf{T} \, \mathbf{r}_B^h
\label{eq:transformacion_homogenea_aplicacion}
\end{equation}

De esta forma, la matriz homogénea permite, en una sola operación matricial,
realizar simultáneamente la rotación y traslación entre dos sistemas de
referencia.  

Entre las propiedades más relevantes de las transformaciones homogéneas se
encuentran (Barrientos et al., \cite{barrientos2014}):
\begin{itemize}
    \item La composición de transformaciones se obtiene mediante el producto
    de matrices homogéneas.
    \item La inversa de una transformación homogénea también es una matriz
    homogénea, lo que facilita el cambio entre sistemas de referencia.
    \item Preservan la estructura geométrica de los puntos y vectores
    transformados.
\end{itemize}

En robótica, las matrices de transformación homogénea constituyen la base para
la formulación de la \gls{cinematica} directa e inversa, permitiendo describir la
posición y orientación del efector final en función de las variables articulares
del manipulador.

\begin{figure}[h]
\centering
\includegraphics[width=0.6\textwidth]{figuras/homogeneas.png}
\caption{Transformación homogénea entre sistemas de referencia. De Barrientos et al. \cite{barrientos2014}.}
\label{fig:Transformacion_homogenea}  
\end{figure}

\subsubsection{\gls{cinematica} de \glspl{robot}}

La \gls{cinematica} de un \gls{robot} estudia el movimiento de su estructura mecánica sin
considerar las fuerzas que lo producen. En el caso de manipuladores
articulados, el análisis cinemático resulta fundamental para establecer la
relación entre las variables articulares (ángulos de las articulaciones o
desplazamientos lineales) y la posición y orientación del efector final en el
espacio de trabajo.

Se distinguen dos problemas principales en la \gls{cinematica} de \glspl{robot}:

\begin{itemize}
    \item \textbf{\gls{cinematica} directa}: consiste en determinar la posición y
    orientación del efector final a partir de los valores de las variables
    articulares. Matemáticamente, puede expresarse como una función
    \begin{equation}
    \mathbf{x} = f(\mathbf{q})
    \label{eq:cinematica_directa}
    \end{equation}
    donde $\mathbf{q}$ es el vector de coordenadas articulares y $\mathbf{x}$
    representa la configuración (posición y orientación) del efector final en
    el espacio cartesiano. Este problema siempre tiene una única solución y se
    resuelve aplicando las matrices de transformación homogénea a lo largo de
    la cadena \gls{cinematica} del \gls{robot}.

    \captionsetup{width=\linewidth}
    \begin{figure}[h]
    \centering
    \includegraphics[width=0.8\textwidth]{figuras/Cinematica.png}
    \caption{Modelo 3D del UR5 y su modelo cinemático correspondiente. De Research Gate et al. \cite{researchgate}.}
    \label{fig:Cinematica_directa}
    \end{figure}

    \item \textbf{\gls{cinematica} inversa}: consiste en determinar los valores de las
    variables articulares necesarios para que el efector final alcance una
    posición y orientación deseadas. Se formula como
    \begin{equation}
    \mathbf{q} = f^{-1}(\mathbf{x})
    \label{eq:cinematica_inversa}
    \end{equation}
    A diferencia de la \gls{cinematica} directa, este problema puede tener múltiples
    soluciones, una solución única o incluso no tener solución, dependiendo de
    la geometría del \gls{robot} y de las restricciones del movimiento.
\end{itemize}

En robótica, la \gls{cinematica} directa es empleada principalmente para simular y
validar el movimiento a partir de valores de entrada conocidos, mientras que la
\gls{cinematica} inversa resulta indispensable en tareas de control y planificación
de trayectorias, ya que permite calcular los comandos articulares necesarios
para alcanzar un objetivo en el espacio de trabajo.

Tal y como señalan Barrientos et al. \cite{barrientos2014}, el dominio de estos
conceptos es esencial en el diseño y control de manipuladores, pues constituye
la base de aplicaciones como la interacción humano–robot, la programación por
demostración y, en el caso del presente proyecto, la comparación entre el
movimiento humano, el ejecutado por el brazo robótico y el registrado por los
instrumentos digitalizadores.

En el presente proyecto, la \gls{cinematica} inversa adquiere una relevancia
especial. Dado que los puntos de referencia del movimiento están definidos por
las coordenadas de una palabra a reproducir, resulta necesario
convertir dichas posiciones cartesianas en valores articulares para el \gls{robot}.
De esta forma, para cada punto de la trayectoria se calculan los ángulos de
las articulaciones, lo que permite ejecutar el movimiento con mayor precisión
y continuidad.  

Además, disponer de los valores articulares posibilita el uso de comandos
específicos de la librería del \gls{robot} orientados a la interpolación en el
espacio articular, lo cual mejora la precisión en la velocidad de ejecución.


\subsubsection{Parámetros de Denavit–Hartenberg}

Para la modelización sistemática de manipuladores robóticos se emplea de forma
habitual la convención de Denavit–Hartenberg (DH), la cual permite describir la
geometría de cada eslabón y articulación mediante un conjunto reducido de
parámetros. Tal y como señalan Barrientos et al. \cite{barrientos2014}, esta
metodología facilita la construcción de las matrices de transformación homogénea
entre eslabones consecutivos, simplificando el análisis cinemático.

En esta convención, cada articulación del \gls{robot} se asocia a un sistema de
coordenadas, y la relación entre dos sistemas consecutivos $\{i-1\}$ y $\{i\}$
se describe mediante cuatro parámetros:

\begin{itemize}
    \item $a_i$: distancia entre los ejes $Z_{i-1}$ y $Z_i$ medida a lo largo de $X_i$.
    \item $\alpha_i$: ángulo entre $Z_{i-1}$ y $Z_i$ alrededor del eje $X_i$.
    \item $d_i$: distancia entre los orígenes de los sistemas medida sobre el eje $Z_{i-1}$.
    \item $\theta_i$: ángulo de rotación alrededor de $Z_{i-1}$ que lleva a alinear $X_{i-1}$ con $X_i$.
\end{itemize}

Con estos parámetros, la matriz de transformación homogénea que relaciona
dos eslabones consecutivos se expresa como:

\begin{equation}
T_i^{i-1} =
\begin{bmatrix}
\cos \theta_i & -\sin \theta_i \cos \alpha_i & \sin \theta_i \sin \alpha_i & a_i \cos \theta_i \\
\sin \theta_i & \cos \theta_i \cos \alpha_i & -\cos \theta_i \sin \alpha_i & a_i \sin \theta_i \\
0 & \sin \alpha_i & \cos \alpha_i & d_i \\
0 & 0 & 0 & 1
\end{bmatrix}
\label{eq:DH_matrix}
\end{equation}

La concatenación de estas transformaciones permite calcular la posición y
orientación del efector final en función de las variables articulares. Este
enfoque constituye la base para la \gls{cinematica} directa y, en consecuencia, para
la resolución de la \gls{cinematica} inversa.

En el contexto del presente proyecto, la parametrización mediante Denavit–
Hartenberg resulta especialmente útil para familiarizarse con la estructura del
brazo robótico empleado y para modelar sus movimientos con precisión. De este
modo, se obtiene una herramienta sistemática para describir matemáticamente
cada articulación y eslabón, lo que facilita la planificación de trayectorias y
la validación de resultados experimentales.


\subsection{Tareas realizadas para cumplir el objeto}

En primer lugar, se llevó a cabo una familiarización con el entorno de trabajo, 
haciendo especial énfasis en el \gls{robot}, su puesta en marcha, la programación básica 
y las medidas de seguridad necesarias para su correcta operación.  

Posteriormente, se desarrolló un programa en \gls{python} para controlar el \gls{robot} \acrfull{UR5} 
mediante la librería oficial \texttt{ur\_rtde} \cite{ur_rtde}. 
El control se realizó utilizando archivos \acrshort{CSV} que contenían las coordenadas de los puntos a dibujar.  

Como ejercicio inicial, se escribieron las primeras 100 palabras del libro \textit{Don Quijote de la Mancha} \cite{cervantes1605},
con la tableta gráfica \acrfull{WIPL}. A continuación, estas mismas palabras se reprodujeron con el \gls{robot} 
sobre la misma tableta, con el fin de comparar los errores de digitalización entre ambos sistemas.  

Una vez obtenidos los datos de las palabras escritas por los tres sistemas (humano, robot y tableta), 
se procedió a su análisis y comparación.  

Adicionalmente, se implementó una \gls{redneuronal} en \gls{matlab} 
para la predicción de señales a partir de diferentes entradas, considerando tres casos:

\begin{itemize}
  \item Predicción del movimiento del robot a partir del movimiento humano (\acrshort{HTR}).
  \item Predicción del movimiento registrado por la tableta a partir del movimiento del robot (\acrshort{RRT}).
  \item Predicción del movimiento registrado por la tableta a partir del movimiento humano (\acrshort{HTRT}).
\end{itemize}

y analizando las metricas de error \acrfull{RMSE}, \acrfull{MAE} y \acrfull{MSE}
para cada caso.

De forma complementaria, también se desarrollaron programas en \gls{matlab} 
para el preprocesamiento de los datos originales, generando un sistema de archivos homogéneo 
que garantizara un formato común y facilitara la comparación entre conjuntos de datos.  

Finalmente, tras el análisis de los resultados obtenidos con la tableta \acrfull{WIPL}, 
se repitió el experimento empleando la tableta \acrfull{IOTP} y un sensor de movimiento 3D, 
ampliando así el alcance del estudio. 

Ademas, se tuvo que desarrollar e imprimir un soporte para los boligrafos de las tabletas,
así como un soporte para el sensor 3D, que permitiera su correcta sujeción al efector final del robot.


\subsection{Normas y referencias}
\subsubsection{Normativa aplicada}
\begin{itemize}
  \item UNE-EN ISO 10218-1:2012. Robots y dispositivos robóticos. Requisitos de seguridad
para robots industriales. Parte 1: Robots. (ISO 10218-1:2011)
  \item UNE-EN ISO 10218-2. Robots y dispositivos robóticos. Requisitos de seguridad para
robots industriales. Parte 2: Sistemas robot e integración. (ISO 10218-2:2011).
  \item UNE-EN ISO 12100:2012. Seguridad de las máquinas. Principios generales 
  para el diseño. Evaluación y reducción del riesgo.
  \item UNE-EN 60204-1:2019. Seguridad de las máquinas. Equipo 
  eléctrico de las máquinas. Parte 1: Requisitos generales.
  \item Real Decreto 842/2002, de 2 de agosto, por el que se aprueba el Reglamento electrotécnico
para baja tensión.
  \item Real Decreto 1215/1997, de 18 de julio, por el que se establecen las disposiciones mínimas
de seguridad y salud para la utilización por los trabajadores de los equipos de
trabajo.
  \item Ley 31/1995, de 8 de noviembre, de Prevención de Riesgos Laborales.

\end{itemize}

% ---- Bibliografía ----
\subsubsection{Referencias bibliográficas}
\printbibliography[heading=none]

\section{Descripción del trabajo realizado}\label{sec:trabajo-realizado}

\subsection{Visión general de la solución adoptada}

La solución adoptada se fundamenta en el desarrollo de una 
plataforma experimental que integra distintos dispositivos de 
captura de escritura y un robot colaborativo \acrshort{UR5}, con 
el objetivo de analizar y modelar los errores introducidos en el 
proceso de digitalización y reproducción de trazos manuscritos. 

El sistema completo se organizó en cuatro fases principales:

\begin{enumerate}
  \item \textbf{Captura de datos.} 
        Se emplearon tres instrumentos distintos: la tableta 
        \acrfull{WIPL}, la tableta prototipo \acrfull{IOTP} y un sistema 
        captor 3D construido a partir 
        de potenciómetros y Arduino. Cada dispositivo registró las 
        trayectorias manuscritas en sus propios formatos de salida, 
        posteriormente unificados.
  \item \textbf{Control del robot.} 
        A través de \gls{python} y la librería \texttt{UR\_RTDE} se 
        estableció comunicación en tiempo real con el 
        \acrshort{UR5}. El robot recibió como entrada las 
        trayectorias previamente capturadas y las reprodujo sobre 
        el instrumento mediante un útil adaptado al extremo de su efector 
        final.
  \item \textbf{Procesamiento y análisis.} 
        En \gls{matlab} se implementaron rutinas de calibración y 
        transformación de coordenadas, asegurando la compatibilidad 
        de datos entre los diferentes instrumentos. Además, se 
        aplicaron técnicas de preprocesado para reducir ruido, 
        homogeneizar longitudes de señales y calcular métricas de 
        error entre las trayectorias humanas y robóticas.
  \item \textbf{Modelado del error.} 
        Se entrenaron redes neuronales con el fin de predecir la 
        señal generada realmente a partir de la referencia humana. 
        Este enfoque permitió evaluar cuantitativamente el error 
        introducido por cada dispositivo y por el propio robot, sin 
        necesidad de ejecutar físicamente todas las pruebas.
\end{enumerate}

Gracias a esta estructura modular, la plataforma permitió 
comparar el comportamiento de distintos dispositivos de captura 
bajo un mismo flujo de trabajo, cuantificando los errores 
introducidos en cada etapa y extrayendo conclusiones sobre la 
fiabilidad y aplicabilidad de los distintos métodos de 
digitalización.

\subsection{Diagrama de flujo del proyecto}

El diagrama de flujo mostrado en la Figura~\ref{fig:Diagrama_flujo} 
resume de manera esquemática el funcionamiento global del sistema 
desarrollado en el TFG. Se distinguen claramente tres bloques 
principales que se corresponden con las fases de captura de datos, 
procesado y análisis, y reproducción de la escritura mediante el 
robot colaborativo.

En la primera fase, los distintos instrumentos digitalizadores 
(\acrshort{WIPL}, \acrshort{IOTP} y sistema captor 3D) registran 
las trazas manuscritas generadas por el usuario. Los datos se 
almacenan en ficheros individuales por palabra, estructurados en 
formato \texttt{CSV}, garantizando una organización homogénea y 
compatible con los entornos de procesado posteriores.

La segunda fase corresponde al procesado de la información. A 
través de scripts desarrollados en MATLAB y Python se lleva a cabo 
la estandarización de los ficheros y la calibración de señales. 


Finalmente, en la tercera fase se realiza la comunicación con el 
robot \acrshort{UR5} mediante la librería \texttt{UR\_RTDE}. Las 
trayectorias capturadas se transmiten al robot, que se encarga de 
reproducirlas directamente sobre el mismo dispositivo utilizado en 
la captura inicial. De este modo, es posible comparar de manera 
directa la escritura humana y la réplica robótica bajo un mismo 
flujo de trabajo.

En conjunto, el diagrama de flujo ilustra la naturaleza modular y 
secuencial de la solución adoptada, donde cada bloque cumple una 
función específica pero integrada en un pipeline común que asegura 
la trazabilidad completa de los datos desde su captura hasta el 
modelado del error.

\begin{figure}[H]
\centering
\includegraphics[scale=0.5,keepaspectratio]{figuras/DiagramaTFG.png}
\caption{Diagrama de flujo del proyecto.}
\label{fig:Diagrama_flujo}
\end{figure}

\subsection{Requisitos de diseño}

Durante la fase inicial del proyecto se establecieron una serie 
de requisitos de diseño que guiaron el desarrollo de la solución 
adoptada. Estos requisitos se clasifican en funcionales y no 
funcionales.

\subsubsection*{Requisitos funcionales}
\begin{itemize}
  \item El sistema debe permitir la captura de escritura manuscrita 
        mediante distintos dispositivos de entrada (Wacom 
        \acrshort{WIPL}, Wacom \acrshort{IOTP} y sistema captor 3D).
  \item Los datos obtenidos deben almacenarse en ficheros 
        estructurados en formato \texttt{CSV}, con un fichero por 
        palabra o ejercicio. Cada fichero debe contener cuatro 
        columnas: Tiempo (s), Coordenada X, Coordenada Y y 
        Coordenada Z.
  \item La base de datos debe organizarse en una estructura jerárquica 
        que distinga el dispositivo de captura empleado 
        (\acrshort{WIPL}, \acrshort{IOTP}, P) y el origen de la señal 
        (HT, R, RT).
  \item El robot colaborativo \acrshort{UR5} debe ser capaz de 
        reproducir las trayectorias capturadas escribiendo 
        directamente sobre el mismo instrumento utilizado en la 
        captura de datos.
  \item El software de control debe gestionar la comunicación en 
        tiempo real con el robot, garantizando la transmisión 
        continua de trayectorias.
  \item El sistema debe incluir un modelo predictivo, basado en 
        redes neuronales, cuyo entrenamiento permita obtener métricas 
        de error (\acrshort{RMSE}, \acrshort{MAE}, \acrshort{MSE}) 
        para comparar la señal real con la predicha y evaluar así la 
        precisión del modelo.
\end{itemize}

\subsubsection*{Requisitos no funcionales}
\begin{itemize}
  \item El sistema debe ser compatible con los formatos de datos 
        generados por los dispositivos empleados (\texttt{CSV}).
  \item El software desarrollado debe ejecutarse en entornos 
        multiplataforma (Python, MATLAB) y ser fácilmente adaptable 
        a otros escenarios de prueba.
  \item El diseño debe respetar los principios de seguridad 
        aplicables a robots colaborativos, siguiendo las normas 
        ISO~10218 e ISO/TS~15066.
  \item La solución debe permitir la trazabilidad completa de los 
        datos, desde la captura inicial hasta la reproducción 
        robótica y el análisis posterior.
  \item El sistema debe mantenerse modular, de forma que sea 
        posible incorporar nuevos dispositivos de captura o 
        algoritmos de análisis sin necesidad de rediseñar la 
        arquitectura completa.
\end{itemize}


\subsection{Hardware y software utilizados}
\subsubsection{Hardware}
\paragraph{Brazo robótico \acrfull{UR5} de Universal Robots.}El \acrshort{UR5} es un robot colaborativo de seis grados de libertad desarrollado 
por Universal Robots, diseñado para tareas de manipulación ligera y programación sencilla \cite{ur5manual}. 
En este proyecto se utilizó como actuador principal encargado de reproducir sobre la tableta 
los mismos trazos previamente capturados de la escritura humana, garantizando la repetibilidad 
y precisión necesarias para el análisis de errores de digitalización.

\begin{figure}[H]
\centering
\includegraphics[width=0.6\textwidth]{figuras/UR5.png}
\caption{Brazo robótico UR5 de Universal Robots de \cite{ur5manual}.}
\label{fig:UR5}
\end{figure}

\paragraph{Tableta gráfica \acrfull{WIPL} de Wacom.}La \acrfull{WIPL} es una tableta digitalizadora profesional desarrollada por Wacom.  
Según el manual de usuario \cite{wacom_manual}, este dispositivo admite entrada tanto con lápiz como táctil, 
dispone de un área activa amplia (311 $\times$ 216 mm en el modelo Large) y está equipada con ocho 
teclas programables \textit{ExpressKeys} y un \textit{Touch Ring} multifunción que facilitan la interacción 
con aplicaciones gráficas \cite{wacom_manual}. El lápiz inalámbrico que la acompaña no requiere batería, 
es sensible a la presión y a la inclinación, lo que permite registrar con precisión la dinámica del trazo 
humano.

En el presente proyecto, la \acrshort{WIPL} se empleó como dispositivo de referencia para la captura 
de escritura manuscrita. Los datos obtenidos se almacenaron en archivos \texttt{CSV} que sirvieron 
posteriormente como entrada para el robot \acrshort{UR5} y para los análisis comparativos. Gracias a 
su resolución y estabilidad, la tableta proporcionó una base fiable para el estudio de errores de 
digitalización y para el entrenamiento de la red neuronal. 

\begin{figure}[H]
\centering
\includegraphics[width=0.5\textwidth]{figuras/WIPL.png}
\caption{Tableta gráfica Wacom Intuos Pro Large de \cite{wacom_manual}.}
\label{fig:Wacom}
\end{figure}

\paragraph{Tableta gráfica \acrfull{IOTP} de Wacom.}La \acrfull{IOTP} es una tableta digitalizadora en fase de prototipo desarrollada por Wacom, 
concebida para investigación académica y aplicaciones educativas. El modelo utilizado en este 
proyecto corresponde a la versión de 10,3 pulgadas, equipada con una pantalla de tinta electrónica 
(\textit{E-Ink Digital Ink Screen Display}) de 297 $\times$ 182 mm, con resolución de 
1872 $\times$ 1404 píxeles (226 dpi) \cite{iotp_manual}.  

La tableta incorpora un lápiz digital \textit{Wacom EMR}, sensible a la presión, a la altura 
y a la inclinación, lo que permite capturar de manera precisa las características dinámicas de la 
escritura manuscrita. Los datos se registran en formato \texttt{InkML}, que incluye coordenadas 
$(x,y)$, presión ejercida, inclinación del lápiz y marcas temporales. Estos ficheros se exportan 
directamente a un ordenador mediante conexión USB Type-C, siendo reconocida por el sistema como 
un dispositivo de almacenamiento masivo. El software proporcionado por la propia empresa genera 
tres tipos de archivos asociados a cada captura: un mapa de bits (\texttt{.bmp}), un fichero 
tabulado (\texttt{.csv}) y un fichero estructurado en \texttt{InkML}. Entre ellos, el formato 
\texttt{.csv} fue el seleccionado para la integración con el robot, ya que permitía transferir 
de manera directa las coordenadas a reproducir en el experimento \cite{iotp_manual}.

En el presente proyecto, la \acrshort{IOTP} se empleó como dispositivo alternativo de 
captura de escritura, con el objetivo de estudiar el error asociado al uso de distintos 
instrumentos de digitalización. Cada tableta, al contar con características técnicas y 
niveles de precisión diferentes, ofrece resultados propios que influyen en el análisis 
final. Los datos obtenidos con la \acrshort{IOTP} se procesaron siguiendo el mismo flujo 
de trabajo que en el caso de la \acrshort{WIPL}, lo que permitió evaluar cómo las 
diferencias entre dispositivos se reflejan en los errores de digitalización y en las 
posibles aplicaciones futuras de este tipo de sistemas.

\begin{figure}[H]
\centering
\includegraphics[width=0.5\textwidth]{figuras/IOTP.png}
\caption{Tableta gráfica Wacom IOTP de \cite{iotp_manual}.}
\label{fig:IOTP}
\end{figure}


\paragraph{Sensor de movimiento 3D mediante potenciometros de 3 ejes.}Además de las tabletas gráficas, se utilizó un sistema de captura de movimiento 
tridimensional diseñado de forma experimental. Este dispositivo estaba compuesto 
por tres potenciómetros conectados a una placa \textit{Arduino}, cuya señal 
permitía estimar la posición de un punto en el espacio tras un proceso de 
calibración previo.  

Para la adquisición de datos se desarrolló un programa en \textbf{Matlab}, que 
se comunicaba directamente con el entorno \textit{Arduino IDE} mediante el puerto 
serie. El código en la placa Arduino se encargaba de leer de forma periódica las 
entradas analógicas asociadas a los potenciómetros y transmitirlas a través del 
puerto serie a alta velocidad. Matlab recibía estas tramas de datos en tiempo real, 
ejecutaba la calibración y procesaba las señales para reconstruir la trayectoria 
espacial realizada.  

Como resultado, el sistema generaba ficheros en formato \texttt{CSV} que contenían 
las coordenadas tridimensionales $(x,y,z)$ de cada punto registrado. Estos datos 
se integraron en el mismo flujo de trabajo que los obtenidos con las tabletas 
gráficas, lo que permitió analizar el error de digitalización no solo en el plano, 
sino también en el eje de profundidad.

\begin{figure}[H]
\centering
\includegraphics[width=0.8\textwidth]{figuras/sistemapotrob.png.jpg}
\caption{Sensor de movimiento 3D mediante potenciometros y robot.}
\label{fig:Sensor3D}
\end{figure}

\paragraph{Ordenador de sobremesa}
\todo{Incluir una tabla con las características técnicas del ordenador.}
\paragraph{Impresora 3D}
\todo{Modelo y características técnicas de la impresora 3D.}

\subsubsection{Software}
En el desarrollo del proyecto se emplearon distintos programas y 
entornos de programación. Cada uno de ellos cumplió una función 
específica tanto en la fase de captura de datos como en la de 
procesado y validación de resultados.

\paragraph{Entorno de desarrollo \acrshort{IDE} PyCharm Community Edition 2021.2.3.}\textit{PyCharm} es un entorno de desarrollo integrado (\acrshort{IDE}) 
especializado en programación con Python. Ofrece herramientas 
avanzadas para la edición de código, depuración, control de 
versiones y gestión de librerías, lo que facilita el desarrollo 
estructurado de proyectos de software. La versión \textit{Community 
Edition} es gratuita y de código abierto, manteniendo la mayoría 
de las funcionalidades necesarias para el ámbito académico.\cite{pycharm}

En el contexto del TFG se empleó como entorno principal para la 
implementación de los programas en Python. PyCharm permitió 
integrar de manera sencilla las librerías externas utilizadas, 
organizar los módulos de código y depurar los scripts encargados 
del procesado de datos y de la comunicación con el robot 
\acrshort{UR5}. 

\begin{figure}[H]
\centering
\includegraphics[width=0.8\textwidth]{figuras/interfazpycarm.png}
\caption{Entorno de desarrollo PyCharm Community Edition 2021.2.3.}
\label{fig:PyCharm}
\end{figure}


\subparagraph{Lenguaje de programación Python 3.12.0.}\textit{Python}
es un lenguaje de programación interpretado, de 
alto nivel y con una sintaxis sencilla, ampliamente extendido en 
ámbitos científicos y de ingeniería. Su diseño orientado a la 
legibilidad y la gran disponibilidad de librerías lo convierten en 
una herramienta idónea para el desarrollo de aplicaciones de 
análisis de datos, inteligencia artificial y control de hardware. \cite{python}

En el marco del TFG se empleó la versión 3.12.0, que permitió 
implementar los módulos encargados de la lectura y procesado de 
los ficheros \texttt{CSV} generados por los dispositivos de 
captura. Asimismo, Python se utilizó como interfaz de 
comunicación con el robot \acrshort{UR5} mediante la librería 
\texttt{UR\_RTDE}, transmitiendo las trayectorias manuscritas al 
manipulador para su reproducción física. 

\subparagraph{Librerías de Python}

\begin{itemize}
  \item \textbf{NumPy}: empleada para cálculos numéricos, 
        manipulación de matrices y operaciones algebraicas 
        necesarias en el tratamiento de señales \cite{numpy}.
  \item \textbf{Pandas}: permitió organizar los datos en 
        estructuras tipo tabla (DataFrames), facilitando la 
        lectura y escritura de archivos CSV \cite{pandas}.
  \item \textbf{Matplotlib}: utilizada para la creación de 
        gráficos y la representación visual de los resultados 
        obtenidos en las pruebas \cite{matplotlib}.
  \item \textbf{UR\_RTDE}: proporcionó la interfaz de 
        comunicación en tiempo real con el robot 
        \acrshort{UR5}, haciendo posible el envío de trayectorias 
        y la monitorización de variables del manipulador \cite{ur_rtde}.
\end{itemize}

\paragraph{Entorno de desarrollo MATLAB R2023b}

\textit{MATLAB} es un entorno de programación y cálculo numérico 
ampliamente utilizado en investigación y desarrollo científico–
técnico. Proporciona un lenguaje propio orientado a matrices, así 
como un extenso conjunto de librerías para el análisis de datos, 
representación gráfica, modelado matemático y diseño de 
algoritmos.\cite{matlab2023} 

En el presente TFG se empleó la versión R2023b, que permitió 
implementar rutinas de preprocesado y calibración de los datos 
capturados por los diferentes instrumentos de escritura. Asimismo, 
MATLAB se utilizó para realizar análisis estadísticos de error y 
para el entrenamiento y validación de la red neuronal destinada a 
evaluar el rendimiento de cada dispositivo. Adicionalmente, para el caso
3D, el entorno sirvió de puente con la placa Arduino, mediante la adquisición en 
tiempo real de los valores analógicos de los potenciómetros, lo 
que facilitó la integración de los datos en el flujo de trabajo 
general del proyecto.

\begin{figure}[H]
\centering
\includegraphics[width=0.8\textwidth]{figuras/interfazMatlab.png}
\caption{Entorno de desarrollo MATLAB R2023b.}
\label{fig:MATLAB}
\end{figure}

\paragraph{Software para captura de \acrshort{WIPL}. HandWriting Capture.}Para la adquisición de datos con la tableta Wacom \acrshort{WIPL} 
se empleó el software \textit{HandWriting Capture V1.0}, 
desarrollado por el Grup de Tractament del Senyal del 
Tecnocampus Mataró Maresme.\cite{HandWritingCapture} 

Esta aplicación permitió registrar de manera directa las 
trazas manuscritas realizadas sobre la tableta y exportarlas en 
archivos \texttt{CSV}, lo que facilitó su posterior integración 
en el flujo de trabajo del proyecto. El programa ofrece una 
interfaz sencilla en la que es posible crear o cargar usuarios, 
ejecutar ejercicios de escritura y almacenar los resultados en 
ficheros de datos. Dichos archivos se guardan automáticamente en 
la carpeta \texttt{data} de la aplicación, garantizando una 
organización homogénea para todas las sesiones de captura.\cite{HandWritingCapture} 

En el marco del TFG, esta herramienta se utilizó como fuente 
principal de datos manuscritos para la tableta \acrshort{WIPL}, 
asegurando la compatibilidad con los entornos de procesado 
implementados en Python y MATLAB.

\begin{figure}[H]
\centering  
\includegraphics[width=0.8\textwidth]{figuras/interfazWIPL.png}
\caption{Software de captura de datos desde la tableta Wacom \acrshort{WIPL}.}
\label{fig:WacomSoftware} 
\end{figure}

\paragraph{Software para captura de \acrshort{IOTP}}La captura de datos de la tableta \acrshort{IOTP} se realizó a 
través del software experimental desarrollado por Wacom y 
distribuido junto con el dispositivo. El programa está basado en 
el protocolo \acrshort{MTP} (\textit{Media Transfer Protocol}), lo 
que permite que la tableta se comporte como un dispositivo de 
almacenamiento externo, similar a una cámara digital o a un 
teléfono móvil. De este modo, los archivos de escritura pueden 
transferirse directamente al ordenador sin necesidad de 
aplicaciones adicionales\cite{iotp_manual}. 

El software genera y gestiona ficheros en formato 
\texttt{InkML}, que contienen tanto las coordenadas $(x,y)$ como 
la información dinámica asociada (presión, inclinación y 
marcas temporales). Además, junto a cada captura se producen 
archivos complementarios en formato \texttt{.bmp} y 
\texttt{.csv}, siendo este último el seleccionado en el marco del 
TFG por su compatibilidad con los entornos de procesado en 
Python y MATLAB \cite{iotp_manual}. 

La aplicación proporciona también herramientas auxiliares, como 
el programa \textit{InkmlConverter}, que permite generar ficheros 
\texttt{InkML} a partir de imágenes de plantilla en mapa de bits, 
para ser utilizadas como fondo de referencia en la escritura. 
Gracias a esta funcionalidad, es posible personalizar los 
ejercicios de captura y mantener un control preciso sobre el 
contenido manuscrito. 

En el contexto del TFG, este software se empleó para registrar, 
almacenar y transferir al ordenador las sesiones de escritura 
realizadas con la \acrshort{IOTP}, integrando posteriormente los 
datos en el flujo de trabajo común junto con los obtenidos de la 
\acrshort{WIPL}.

\begin{figure}[H]
\centering
\includegraphics[width=0.8\textwidth]{figuras/interfazIOTP.png}
\caption{Software de captura de datos desde la tableta Wacom \acrshort{IOTP}.}
\label{fig:IOTPSoftware}
\end{figure}


\paragraph{Software Arduino IDE.}El \textit{Arduino IDE} es el entorno de desarrollo oficial para la 
programación de placas basadas en microcontroladores de la 
plataforma Arduino. Este software permite escribir, compilar y 
cargar programas (denominados \textit{sketches}) en la placa a 
través de una interfaz sencilla y multiplataforma. Además, 
incluye un monitor serie que facilita la comunicación en tiempo 
real entre el microcontrolador y el ordenador. 

En el marco del TFG, el Arduino IDE se utilizó para desarrollar y 
transferir el código encargado de la lectura analógica de los tres 
potenciómetros del sistema captor 3D. Los valores obtenidos se 
enviaban de forma continua por el puerto serie, de modo que 
pudieran ser adquiridos posteriormente en MATLAB para su 
procesado, calibración y análisis. 


\begin{figure}[H]
\centering  
\includegraphics[width=0.8\textwidth]{figuras/interfazIDEArduino.png}
\caption{Entorno de desarrollo Arduino IDE.}
\label{fig:ArduinoIDE}
\end{figure}

\subsection{Dificultades encontradas y soluciones adoptadas}

Durante el desarrollo del proyecto se identificaron diferentes 
dificultades relacionadas con la captura y el análisis de datos. 
A continuación, se describen las más relevantes junto con las 
medidas adoptadas para su resolución, incorporando ejemplos 
gráficos que ilustran el efecto de cada problema.

\subsubsection*{Errores por separación excesiva del lápiz}
En la captura realizada con la tableta \acrshort{WIPL} se 
observó que, cuando el lápiz se separaba demasiado de la 
superficie, la tableta dejaba de registrar puntos intermedios. 
Esto provocaba saltos bruscos en las coordenadas registradas y 
picos anómalos en la señal de velocidad, que no correspondían a 
un movimiento real de escritura. Esto se traducia en movimientos bruscos 
del robot pasando de una coordenada a otra sin suavizado, 
lo que afectaba a la calidad de la reproducción.
La solución adoptada consistió en mantener el lápiz siempre 
próximo a la superficie durante el ejercicio, incluso en los 
movimientos de transición sin contacto directo, evitando así 
pérdidas de muestreo.

\begin{figure}[H]
  \centering
  \includegraphics[width=0.9\textwidth]{figuras/prueba_picovelocidad.png}
  \caption{Ejemplo de pico de velocidad generado por ausencia de 
           puntos intermedios al separar demasiado el lápiz de la 
           tableta.}
  \label{fig:picoVelocidad}
\end{figure}

\subsubsection*{Errores por activación involuntaria de botones}
Otro problema detectado estuvo asociado al lápiz digital de la 
\acrshort{WIPL}. Durante la escritura, la pulsación accidental 
de los botones laterales provocaba la aparición de valores 
anómalos en la coordenada Z (valores distintos de 0 o 1), lo que 
se traducía en distorsiones en la señal y nuevos picos de 
velocidad.  


\begin{figure}[H]
  \centering
  \includegraphics[width=0.5\textwidth]{figuras/boligrafoWacom.png}
  \caption{Lápiz digital de la tableta Wacom \acrshort{WIPL}, con los
           botones laterales que pueden activarse 
           involuntariamente durante la escritura.}
  \label{fig:boligrafoWacom}
\end{figure}

La solución consistió en evitar la activación de los botones 
durante las sesiones experimentales y, en caso necesario, 
deshabilitar sus funciones en la configuración del dispositivo.


\begin{figure}[H]
  \centering
  \includegraphics[width=0.7\textwidth]{figuras/z3.png}
  \caption{Valores anómalos de la coordenada Z asociados a la 
           activación involuntaria de los botones del lápiz, que 
           generan distorsiones en la señal de velocidad.}
  \label{fig:Z3}
\end{figure}

\subsubsection*{Desfases entre señales humanas y robóticas}
Durante la comparación de trayectorias \acrlong{R} y \acrlong{RT} 
se identificaron desfases tanto espaciales como temporales. Este 
problema dificultaba la superposición directa de señales, lo que 
impedía un análisis cuantitativo inmediato.  

Para abordar esta incidencia se implementaron técnicas de 
preprocesado en MATLAB. En particular, se aplicó el centrado de 
trayectorias mediante el cálculo del centroide, con el fin de 
reducir desplazamientos espaciales iniciales, y se utilizó 
interpolación temporal para homogeneizar la longitud de las 
señales. Ambas correcciones se mantuvieron dentro del flujo de 
trabajo, dado que no modificaban sustancialmente el error real de 
los dispositivos y mejoraban la comparabilidad de las trayectorias.  

En cambio, la técnica de alineación temporal \textit{Dynamic Time 
Warping} (DTW), aunque inicialmente considerada, fue descartada. 
Su aplicación implicaba alterar la dinámica original de las 
señales, lo que hubiera supuesto perturbar los errores reales que 
se pretendían analizar. Por este motivo, se decidió no emplearla y 
trasladar el tratamiento de los desfases restantes al modelado con 
redes neuronales, permitiendo que el modelo aprendiera de manera 
natural las discrepancias presentes entre trayectorias.

\begin{figure}[H]
  \centering
  \includegraphics[width=0.9\textwidth]{figuras/Comparacion Ejes.png}
  \caption{Ejemplo de desfase entre la trayectoria \acrshort{R} 
          y la trayectoria \acrshort{RT}.}
\end{figure}

\subsubsection*{Velocidad excesiva en la captura 3D}
En el caso del sistema captor 3D, la primera toma de datos 
realizada por el usuario se llevó a cabo con una velocidad de 
ejecución demasiado elevada. Al reproducirse en el robot, esta 
dinámica suponía un riesgo para la integridad del sistema de 
sujeción y del propio sistema captor 3D, ya que los 
movimientos eran más rápidos de lo que el montaje podía tolerar.  

Ante este problema se decidió repetir la toma de datos, esta vez 
con una escritura más lenta y precisa por parte del usuario. De 
este modo, el robot pudo reproducir las trayectorias sin poner en 
riesgo los elementos mecánicos ni comprometer la estabilidad del 
experimento.

En conjunto, la resolución de estas incidencias permitió obtener 
un flujo de datos más robusto y representativo, garantizando que 
los errores analizados fuesen intrínsecos al sistema y no fruto de 
artefactos de captura o preprocesado.

\subsection{Disposición física de los elementos}

La Figura~\ref{fig:disposicionfisica} muestra la disposición 
física y las conexiones entre los diferentes componentes del 
sistema experimental. El ordenador actúa como unidad central de 
procesamiento, ejecutando los programas desarrollados en Python y 
MATLAB para la lectura de datos, el preprocesado de ficheros y el 
envío de trayectorias al robot. 

El ordenador se conecta al controlador del robot \acrshort{UR5} a 
través de un cable Ethernet, lo que permite la comunicación en 
tiempo real mediante la librería \texttt{UR\_RTDE}. De manera 
paralela, el instrumento digitalizador empleado en cada experimento 
(\acrshort{WIPL}, \acrshort{IOTP} o sistema captor 3D) se conecta 
al ordenador por USB, asegurando la adquisición directa de los 
datos manuscritos en formato digital. 

El controlador del robot está vinculado físicamente al 
\textit{teach pendant} (interfaz manual), que permite ejecutar 
operaciones básicas de configuración y seguridad. Finalmente, el 
robot recibe las trayectorias procesadas desde el ordenador y las 
reproduce directamente sobre el instrumento digitalizador, 
garantizando así la coherencia entre la escritura humana y la 
réplica robótica. 

Esta disposición modular asegura la trazabilidad de los datos, 
desde la captura inicial hasta la reproducción física de las 
trayectorias, y facilita la sustitución de instrumentos de 
captura sin necesidad de modificar la arquitectura general del 
sistema.

\begin{figure}[H]
\centering
\includegraphics[width=0.9\textwidth]{figuras/disposicionfisica.png}
\caption{Esquema de la disposición física del sistema.}
\label{fig:disposicionfisica}
\end{figure}

\begin{figure}[H]
\centering
\includegraphics[width=0.7\textwidth]{figuras/disposicionreal.png}
\caption{Disposición real del sistema en el laboratorio.}
\label{fig:disposicionreal}
\end{figure}

\subsection{Descripción detallada de la solución}
\todo{TODO lo gordo}
\todo{Sacar foto de la configuracion manual de las herramientas del robot en el teach pendant}
\todo{Poner todo esto en la parte de 3D}

\begin{figure}[H]
\centering
\begin{tikzpicture}[node distance=18mm]
\tikzset{
  block/.style   = {rectangle, rounded corners, minimum width=4cm, minimum height=1cm,
                    text centered, draw=black, fill=blue!20},
  process/.style = {rectangle, minimum width=4cm, minimum height=1cm,
                    text centered, draw=black, fill=green!20},
  data/.style = {trapezium, trapezium left angle=70, trapezium right angle=70,
               minimum width=4cm, minimum height=1cm, text centered, draw=black, fill=orange!20},
  arrow/.style   = {thick, -{Stealth[length=3mm,width=2mm]}}
}

\node (pots)    [block]                   {Potenciómetros (x3)};
\node (arduino) [process, below of=pots]  {Placa Arduino - Lectura analógica + envío serie};
\node (matlab)  [process, below of=arduino] {Programa Matlab - Recepción + Calibración + Procesado};
\node (csv)     [data,    below of=matlab] {Archivo CSV (x,y,z)};

\draw[arrow] (pots) -- (arduino);
\draw[arrow] (arduino) -- node[right]{\small Puerto serie} (matlab);
\draw[arrow] (matlab) -- (csv);
\end{tikzpicture}
\caption{Flujo de funcionamiento del sistema captor 3D (vertical).}
\end{figure}

\begin{figure}[H]
\centering
\begin{forest}
  for tree={
     font=\ttfamily,
     grow'=south,
     child anchor=north,
     parent anchor=south,
     anchor=west,
     edge={ultra thin},
     l sep=8pt,
     s sep=4pt
    }
 [Base de Datos
     [WIPL
         [HT]
         [R]
         [RT]
     ]
     [IOTP
         [HT]
         [R]
         [RT]
     ]
     [P
         [HP]
         [R]
         [RP]
     ]
  ]
\end{forest}
\caption{Estructura jerárquica de los directorios de la base de datos}
\label{fig:estructuraDirectorios}
\end{figure}


\begin{table}[h]
			\centering
			\begin{tabular}{|c|c|c|c|}
				\hline
				\textbf{Tiempo (s)} & \textbf{Coordenada X} & \textbf{Coordenada Y} & \textbf{Coordenada Z} \\
				\hline
				0.000 & 12.345 & 67.890 & 0.000 \\
				0.008 & 12.350 & 67.895 & 0.000 \\
				0.016 & 12.360 & 67.910 & 1.000 \\
				\vdots & \vdots & \vdots & \vdots \\
				\hline
			\end{tabular}
			\caption{Estructura de los ficheros de datos por palabra.}
            \label{cuad:ArquitecturaDatos}
\end{table}



\subsection{Resultados}
\subsubsection{Metodología experimental}
\pendiente

\subsubsection{Aplicación del modelo entrenado}
\pendiente

\subsubsection{Resultados obtenidos en metricas clave}


\subsection{Conclusiones finales}
\pendiente

\subsection{Posibles mejoras y trabajo futuro}
\pendiente




% ==============================================
% DOCUMENTO II: MEMORIA DE CÁLCULO
% ==============================================
\chapter{Documento II: Memoria de Cálculo}

\addcontentsline{toc}{chapter}{Documento II: Memoria de Cálculo}
\section{Hipótesis, condiciones de contorno y criterios de diseño}
\section{Modelos matemáticos y ecuaciones empleadas}
\section{Procedimientos de cálculo y dimensionamiento}
\section{Resultados de cálculo (tablas y figuras)}
\section{Verificación, validación y análisis de incertidumbre}
\section{Conclusiones de la memoria de cálculo}






\clearpage

% ==============================================
% DOCUMENTO III: PLIEGO DE CONDICIONES TÉCNICAS
% ==============================================
\chapter{Documento III: Pliego de Condiciones Técnicas}
\addcontentsline{toc}{chapter}{Documento III: Pliego de Condiciones Técnicas}
\section{Condiciones generales}
\section{Clasificación de la instalación}
\section{Criterios de diseño y ejecución}
\section{Materiales y equipos}
\section{Condiciones de montaje}
\section{Programa de mantenimiento}
\section{Pruebas y puesta en servicio}
\section{Documentación necesaria}





\clearpage

% ==============================================
% DOCUMENTO IV: PRESUPUESTO
% ==============================================
\chapter{Documento IV: Presupuesto}


\addcontentsline{toc}{chapter}{Documento IV: Presupuesto}
\section{Cuadro de precios unitarios}
\section{Medición y valoración}
\section{Resumen por capítulos}
\section{Presupuesto de ejecución material}
\section{Costes indirectos y gastos generales}
\section{Presupuesto total}




\clearpage

% ==============================================
% DOCUMENTO V: ESQUEMAS (CONEXIONADO DE BLOQUES)
% ==============================================
\chapter{Documento V: Esquemas y Conexionados}
\addcontentsline{toc}{chapter}{Documento V: Esquemas y Conexionados}
\section{Diagrama de bloques del sistema}
\section{Esquemas de comunicaciones / red}

\clearpage
% ==============================================
% DOCUMENTO VI: ANEXOS
% ==============================================
\chapter{Documento VI: Anexos}



\addcontentsline{toc}{chapter}{Documento VI: Anexos}
\section{Listados de código y scripts}
\todo{Codigo fuente del control del robot en Python.}
\todo{Codigo fuente del preprocesamiento de datos en Matlab para cada caso.}
\todo{Codigo de Arduino IDE para el sensor 3D.}
\todo{Codigo fuente de la red neuronal en Matlab.}
\section{Manuales / datasheets relevantes}




\end{document}

